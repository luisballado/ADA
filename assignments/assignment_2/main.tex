%%%%%%%%%%%%%%%%%%%%%%%%%%%%%%%%%%%%%%%%%
% Lachaise Assignment
% LaTeX Template
% Version 1.0 (26/6/2018)
%
% This template originates from:
% http://www.LaTeXTemplates.com
%
% Authors:
% Marion Lachaise & François Févotte
% Vel (vel@LaTeXTemplates.com)
%
% License:
% CC BY-NC-SA 3.0 (http://creativecommons.org/licenses/by-nc-sa/3.0/)
% 
%%%%%%%%%%%%%%%%%%%%%%%%%%%%%%%%%%%%%%%%%

%----------------------------------------------------------------------------------------
%	PACKAGES AND OTHER DOCUMENT CONFIGURATIONS
%----------------------------------------------------------------------------------------

\documentclass{article}

\input{structure.tex} % Include the file specifying the document structure and custom commands

%----------------------------------------------------------------------------------------
%	ASSIGNMENT INFORMATION
%----------------------------------------------------------------------------------------

\title{ITC-ADA-C1-2023: Assignment \#2} % Title of the assignment

\author{Luis Ballado\\ \texttt{luis.ballado@cinvestav.mx}} % Author name and email address

\date{CINVESTAV UNIDAD TAMAULIPAS --- \today} % University, school and/or department name(s) and a date

%----------------------------------------------------------------------------------------

\begin{document}

\maketitle % Print the title

%----------------------------------------------------------------------------------------
%	INTRODUCTION
%----------------------------------------------------------------------------------------

\section{Para los siguientes pares de funciones indique si la primera tiene un orden de crecimiento menor, mayor o igual al de la segunda}

\begin{itemize}
\item $n(n+1)$ y $200n^{2}$
\item $100n^{2}$ y $0,01n^{3}$
\item $log_{2}n$ y $ln n$
\item $log_{2} n$ y $log_{2}n^{2}$
\item $2^{n-1}$ y $2^{n}$
\item $(n-1)!$ y $n!$
\end{itemize}

\newpage
\section{Compare el orden de crecimiento de los siguientes pares de fuciones empleando límites e indique el resultado usando la notación adecuada ($O$,$\Omega$,$\Theta$,etc).}

\begin{itemize}
\item $n!$ y $2^{n}$
\item $n^{3}$ y $2^{n}$
\end{itemize}

%----------------------------------------------------------------------------------------
\newpage
\section{Calcule las siguientes sumatorias describiendo en su respuesta las propiedades que emplea en cada paso:}

\begin{itemize}
\item 
\item
\item 
\item 
\end{itemize}

\section{Considere el algoritmo mostrado y responda las siguientes preguntas:}

\begin{itemize}
\item ¿Qué calcula el algoritmo?
\item ¿Cuál es la operación básica?
\item ¿Cuántas veces se ejecuta la operación básica?
\item ¿Cuál es la clase de eficiencia a la que petenece este algoritmo?
\item Sugiera una mejora, o un mejor algoritmo e indique su clase de eficiencia. Si esto no es posible, pruébelo.
\end{itemize}

\section{En un máximo de dos párrafos comente sus conclusiones personales acerca de esta tarea. Podría en ellas tocar algunos de los siguientes puntos: qué aprendí o, qué piensa acerca del uso de estos conceptos matemáticos para analizar algoritmos, qué se le facilitó(dificultó) más al hacer esta tarea, comente un potencial ejemplo de cómo usaría estos conceptos para analizar un algoritmo visto en otro curso, etc.}

\end{document}

