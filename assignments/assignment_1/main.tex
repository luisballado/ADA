%%%%%%%%%%%%%%%%%%%%%%%%%%%%%%%%%%%%%%%%%
% Lachaise Assignment
% LaTeX Template
% Version 1.0 (26/6/2018)
%
% This template originates from:
% http://www.LaTeXTemplates.com
%
% Authors:
% Marion Lachaise & François Févotte
% Vel (vel@LaTeXTemplates.com)
%
% License:
% CC BY-NC-SA 3.0 (http://creativecommons.org/licenses/by-nc-sa/3.0/)
% 
%%%%%%%%%%%%%%%%%%%%%%%%%%%%%%%%%%%%%%%%%

%----------------------------------------------------------------------------------------
%	PACKAGES AND OTHER DOCUMENT CONFIGURATIONS
%----------------------------------------------------------------------------------------

\documentclass{article}

%%%%%%%%%%%%%%%%%%%%%%%%%%%%%%%%%%%%%%%%%
% Lachaise Assignment
% Structure Specification File
% Version 1.0 (26/6/2018)
%
% This template originates from:
% http://www.LaTeXTemplates.com
%
% Authors:
% Marion Lachaise & François Févotte
% Vel (vel@LaTeXTemplates.com)
%
% License:
% CC BY-NC-SA 3.0 (http://creativecommons.org/licenses/by-nc-sa/3.0/)
% 
%%%%%%%%%%%%%%%%%%%%%%%%%%%%%%%%%%%%%%%%%

%----------------------------------------------------------------------------------------
%	PACKAGES AND OTHER DOCUMENT CONFIGURATIONS
%----------------------------------------------------------------------------------------

\usepackage{amsmath,amsfonts,stmaryrd,amssymb} % Math packages
\usepackage[dvipsnames]{xcolor}
\usepackage{enumerate} % Custom item numbers for enumerations
\usepackage{hyperref}
\usepackage[ruled,vlined]{algorithm2e} % Algorithms

\usepackage[framemethod=tikz]{mdframed} % Allows defining custom boxed/framed environments

\usepackage{listings} % Required for insertion of code

\newcommand{\randomcolor}{%
  \definecolor{randomcolor}{RGB}
   {
    \pdfuniformdeviate 256,
    \pdfuniformdeviate 256,
    \pdfuniformdeviate 256
   }%
  \color{randomcolor}%
}

\definecolor{codegreen}{rgb}{0,0.6,0}
\definecolor{codegray}{rgb}{0.5,0.5,0.5}
\definecolor{codepurple}{rgb}{0.58,0,0.82}
\definecolor{backcolour}{rgb}{1,1,1}
\lstdefinestyle{mystyle}{
    backgroundcolor=\color{backcolour},   
    commentstyle=\color{codegreen},
    keywordstyle=\color{magenta},
    numberstyle=\tiny\color{codegray},
    stringstyle=\color{codepurple},
    basicstyle=\ttfamily\footnotesize,
    breakatwhitespace=false,         
    breaklines=true,                 
    captionpos=b,                    
    keepspaces=true,                 
    numbers=left,                    
    numbersep=5pt,                  
    showspaces=false,                
    showstringspaces=false,
    showtabs=false,                  
    tabsize=2
}
\renewcommand{\lstlistingname}{Código}% Listing -> Algorithm
\lstset{style=mystyle}


%\usepackage{listings} % File listings, with syntax highlighting
%\lstset{
%	basicstyle=\ttfamily, % Typeset listings in monospace font
%}

%----------------------------------------------------------------------------------------
%	DOCUMENT MARGINS
%----------------------------------------------------------------------------------------

\usepackage{geometry} % Required for adjusting page dimensions and margins

\geometry{
	paper=a4paper, % Paper size, change to letterpaper for US letter size
	top=2.5cm, % Top margin
	bottom=3cm, % Bottom margin
	left=2.5cm, % Left margin
	right=2.5cm, % Right margin
	headheight=14pt, % Header height
	footskip=1.5cm, % Space from the bottom margin to the baseline of the footer
	headsep=1.2cm, % Space from the top margin to the baseline of the header
	%showframe, % Uncomment to show how the type block is set on the page
}

%----------------------------------------------------------------------------------------
%	FONTS
%----------------------------------------------------------------------------------------

\usepackage[utf8]{inputenc} % Required for inputting international characters
\usepackage[T1]{fontenc} % Output font encoding for international characters

\usepackage{XCharter} % Use the XCharter fonts
\usepackage{pgfplots}
\usepackage{multicol}
%----------------------------------------------------------------------------------------
%	COMMAND LINE ENVIRONMENT
%----------------------------------------------------------------------------------------

% Usage:
% \begin{commandline}
%	\begin{verbatim}
%		$ ls
%		
%		Applications	Desktop	...
%	\end{verbatim}
% \end{commandline}

\mdfdefinestyle{commandline}{
	leftmargin=10pt,
	rightmargin=10pt,
	innerleftmargin=15pt,
	middlelinecolor=black!50!white,
	middlelinewidth=2pt,
	frametitlerule=false,
	backgroundcolor=black!5!white,
	frametitle={Command Line},
	frametitlefont={\normalfont\sffamily\color{white}\hspace{-1em}},
	frametitlebackgroundcolor=black!50!white,
	nobreak,
}

% Define a custom environment for command-line snapshots
\newenvironment{commandline}{
	\medskip
	\begin{mdframed}[style=commandline]
}{
	\end{mdframed}
	\medskip
}

%----------------------------------------------------------------------------------------
%	FILE CONTENTS ENVIRONMENT
%----------------------------------------------------------------------------------------

% Usage:
% \begin{file}[optional filename, defaults to "File"]
%	File contents, for example, with a listings environment
% \end{file}

\mdfdefinestyle{file}{
	innertopmargin=1.6\baselineskip,
	innerbottommargin=0.8\baselineskip,
	topline=false, bottomline=false,
	leftline=false, rightline=false,
	leftmargin=2cm,
	rightmargin=2cm,
	singleextra={%
		\draw[fill=black!10!white](P)++(0,-1.2em)rectangle(P-|O);
		\node[anchor=north west]
		at(P-|O){\ttfamily\mdfilename};
		%
		\def\l{3em}
		\draw(O-|P)++(-\l,0)--++(\l,\l)--(P)--(P-|O)--(O)--cycle;
		\draw(O-|P)++(-\l,0)--++(0,\l)--++(\l,0);
	},
	nobreak,
}

% Define a custom environment for file contents
\newenvironment{file}[1][File]{ % Set the default filename to "File"
	\medskip
	\newcommand{\mdfilename}{#1}
	\begin{mdframed}[style=file]
}{
	\end{mdframed}
	\medskip
}

%----------------------------------------------------------------------------------------
%	NUMBERED QUESTIONS ENVIRONMENT
%----------------------------------------------------------------------------------------

% Usage:
% \begin{question}[optional title]
%	Question contents
% \end{question}

\mdfdefinestyle{question}{
	innertopmargin=1.2\baselineskip,
	innerbottommargin=0.8\baselineskip,
	roundcorner=5pt,
	nobreak,
	singleextra={%
		\draw(P-|O)node[xshift=1em,anchor=west,fill=white,draw,rounded corners=5pt]{%
		Pregunta \theQuestion\questionTitle};
	},
}

\newcounter{Question} % Stores the current question number that gets iterated with each new question

% Define a custom environment for numbered questions
\newenvironment{question}[1][\unskip]{
	\bigskip
	\stepcounter{Question}
	\newcommand{\questionTitle}{~#1}
	\begin{mdframed}[style=question]
}{
	\end{mdframed}
	\medskip
}

%----------------------------------------------------------------------------------------
%	WARNING TEXT ENVIRONMENT
%----------------------------------------------------------------------------------------

% Usage:
% \begin{warn}[optional title, defaults to "Warning:"]
%	Contents
% \end{warn}

\mdfdefinestyle{warning}{
	topline=false, bottomline=false,
	leftline=false, rightline=false,
	nobreak,
	singleextra={%
		\draw(P-|O)++(-0.5em,0)node(tmp1){};
		\draw(P-|O)++(0.5em,0)node(tmp2){};
		\fill[black,rotate around={45:(P-|O)}](tmp1)rectangle(tmp2);
		\node at(P-|O){\color{white}\scriptsize\bf !};
		\draw[very thick](P-|O)++(0,-1em)--(O);%--(O-|P);
	}
}

% Define a custom environment for warning text
\newenvironment{warn}[1][Warning:]{ % Set the default warning to "Warning:"
	\medskip
	\begin{mdframed}[style=warning]
		\noindent{\textbf{#1}}
}{
	\end{mdframed}
}

%----------------------------------------------------------------------------------------
%	INFORMATION ENVIRONMENT
%----------------------------------------------------------------------------------------

% Usage:
% \begin{info}[optional title, defaults to "Info:"]
% 	contents
% 	\end{info}

\mdfdefinestyle{info}{%
	topline=false, bottomline=false,
	leftline=false, rightline=false,
	nobreak,
	singleextra={%
		\fill[black](P-|O)circle[radius=0.4em];
		\node at(P-|O){\color{white}\scriptsize\bf i};
		\draw[very thick](P-|O)++(0,-0.8em)--(O);%--(O-|P);
	}
}

% Define a custom environment for information
\newenvironment{info}[1][Info:]{ % Set the default title to "Info:"
	\medskip
	\begin{mdframed}[style=info]
		\noindent{\textbf{#1}}
}{
	\end{mdframed}
}
 % Include the file specifying the document structure and custom commands

%----------------------------------------------------------------------------------------
%	ASSIGNMENT INFORMATION
%----------------------------------------------------------------------------------------

\title{ITC-ADA-C1-2023: Assignment \#1} % Title of the assignment

\author{Luis Ballado\\ \texttt{luis.ballado@cinvestav.mx}} % Author name and email address

\date{CINVESTAV UNIDAD TAMAULIPAS --- \today} % University, school and/or department name(s) and a date

%----------------------------------------------------------------------------------------

\begin{document}

\maketitle % Print the title

%----------------------------------------------------------------------------------------
%	INTRODUCTION
%----------------------------------------------------------------------------------------

\section{Diseñe un algoritmo para encontrar todos los elementos comunes en dos listas ordenadas de números}

\begin{info} % Information block
  Se hace la prueba con los siguientes valores\\
  $2,5,5,5$ y $2,2,3,5,5,7$
\end{info}

\begin{question}
  \textbf{¿Cuál es el número máximo de comparaciones que realiza su algoritmo en función de las longitudes de las listas (m y n) respectivamente?}

  %se responde aquí
  \textit{El número máximo de comparaciones viene dado por la busqueda de los elementos del primer arreglo con el segundo, es decir aplicando la regla del producto m x n, pero se puede reducir con los elementos repetidos y reducir las comparaciones}
    
\end{question}

%------------------------------------------------

\subsection{Pseudocodigo}

\begin{center}
  \begin{minipage}{0.7\linewidth} % Adjust the minipage width to accomodate for the length of algorithm lines
    \begin{algorithm}[H] 
      \SetKwInOut{Input}{entrada}\SetKwInOut{Output}{salida}
      \Input{$Arreglo_{1}, Arreglo_{2}$}
      \Output{$Arreglo_{3}$ (arreglo con repetidos)}
      \DontPrintSemicolon
      
      \caption{Encontrar elementos iguales}
      \label{alg:loop}
            {$CalcularLongitud(arreglo_{1})$}\\
            {$CalcularLongitud(arreglo_{2})$}

            \For{$i \gets 0$ hasta $longitud(arreglo_{1})$} {
              \For{$j \gets 0$ hasta $longitud(arreglo_{2})$} {
                \If{$arreglo_{1}[i]==arreglo_{2}[j]$}{
                  \If{$!Existe(arr,arreglo_{1}[i]$)}{
                    $arreglo_{3}.agregar(a[i])$\;
                  }
                }
                break;\;
              }
            }
    \end{algorithm}
  \end{minipage}
\end{center}

\begin{center}
  \begin{minipage}{0.7\linewidth} % Adjust the minipage width to accomodate for the length of algorithm lines
    \begin{algorithm}[H] 
      \SetKwInOut{Input}{input}\SetKwInOut{Output}{output}
      \Input{$arreglo_{3},valor$}
      \Output{$verdadero$ ó $falso$ (booleano si existe el elemento ó no)}
      \DontPrintSemicolon
      \caption{Función Existe}
      \label{alg:loop}
      \SetKwFunction{FLoop}{Existe}
      \SetKwProg{Fn}{Function}{}{}
      \BlankLine
      \Fn{\FLoop{$arreglo_{3},valor$}}{
        \For{$i \gets 0$ hasta $longitud(arreglo_{3})$} {
          \If{$arreglo_{3}[i]==valor$}{
            \KwRet {$Verdadero$}
          }
        }
        \KwRet {$Falso$}
      }
    \end{algorithm}
  \end{minipage}
\end{center}

%------------------------------------------------

\newpage
\subsection{Implementación}

% File contents
\begin{file}[tarea1.cpp]
\begin{lstlisting}[language=C++]
#include <iostream>
#include <vector>

bool existe(std::vector<int> _vector_, int valor){
  for(int i = 0; i < _vector_.size(); i++){ // n
    if (_vector_[i] == valor){              // 1
      return true;
    }
  }
  return false;
}

//Complejidad funcion principal O(n^2)
int main(){

  int a[] = {2,5,5,5};                     // 1
  int b[] = {2,2,3,5,5,7};                 // 1

  //vector de resultados
  std::vector<int>arr;                     // 1

  //longitud array1
  int a_length = sizeof(a)/sizeof(a[0]);   // 1
  //longitud array2
  int b_length = sizeof(b)/sizeof(b[0]);   // 1
  
  //Usando fuerza bruta                    // O(n^2)
  for (int i = 0; i < a_length; i++){      
    for (int j = 0; j < b_length; j++){    
      if (a[i] == b[j]){                   // 1
        //revisar si ya existe en el array
        if (!existe(arr,a[i])){            // 1
          arr.push_back(a[i]);
        }
        break;
      }
    }
  }

  // Imprimir resultado
  std::cout << "El resultado es: \n";       // 1
  for(int i = 0; i<arr.size(); i++){        // n
    std::cout << arr[i] << "\n";            // 1
  } 
}
\end{lstlisting}
\end{file}

Ejecutar desde una terminal

% Command-line "screenshot"
\begin{commandline}
	\begin{verbatim}
		$ g++ -o ./tarea1 ./tarea1.cpp
		$ ./tarea1
	\end{verbatim}
\end{commandline}



%----------------------------------------------------------------------------------------
\newpage
\subsection{Análisis de complejidad del mejor y peor caso}
\begin{warn}[MEJOR CASO:]
  encontrar todos los elementos del primer arreglo en el segundo arreglo de forma secuencial, sin el analisis de numeros que pertenezcan al primer arreglo
\end{warn}
\begin{warn}[PEOR CASO:]
  analizar todas las posibilidades y no encontrar similitudes $O(n^{2})$
\end{warn}

\end{document}

