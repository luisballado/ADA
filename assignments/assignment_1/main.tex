%%%%%%%%%%%%%%%%%%%%%%%%%%%%%%%%%%%%%%%%%
% Lachaise Assignment
% LaTeX Template
% Version 1.0 (26/6/2018)
%
% This template originates from:
% http://www.LaTeXTemplates.com
%
% Authors:
% Marion Lachaise & François Févotte
% Vel (vel@LaTeXTemplates.com)
%
% License:
% CC BY-NC-SA 3.0 (http://creativecommons.org/licenses/by-nc-sa/3.0/)
% 
%%%%%%%%%%%%%%%%%%%%%%%%%%%%%%%%%%%%%%%%%

%----------------------------------------------------------------------------------------
%	PACKAGES AND OTHER DOCUMENT CONFIGURATIONS
%----------------------------------------------------------------------------------------

\documentclass{article}

\input{structure.tex} % Include the file specifying the document structure and custom commands

%----------------------------------------------------------------------------------------
%	ASSIGNMENT INFORMATION
%----------------------------------------------------------------------------------------

\title{ITC-ADA-C1-2023: Assignment \#1} % Title of the assignment

\author{Luis Ballado\\ \texttt{luis.ballado@cinvestav.mx}} % Author name and email address

\date{CINVESTAV UNIDAD TAMAULIPAS --- \today} % University, school and/or department name(s) and a date

%----------------------------------------------------------------------------------------

\begin{document}

\maketitle % Print the title

%----------------------------------------------------------------------------------------
%	INTRODUCTION
%----------------------------------------------------------------------------------------

\section{Diseñe un algoritmo para encontrar todos los elementos comunes en dos listas ordenadas de números}

\begin{info} % Information block
  Se hace la prueba con los siguientes valores\\
  $2,5,5,5$ y $2,2,3,5,5,7$
\end{info}

\begin{question}
  \textbf{¿Cuál es el número máximo de comparaciones que realiza su algoritmo en función de las longitudes de las listas (m y n) respectivamente?}

  %se responde aquí
  \textit{El número máximo de comparaciones viene dado por la busqueda de los elementos del primer arreglo con el segundo, es decir aplicando la regla del producto m x n, pero se puede reducir con los elementos repetidos ó con el manejo de un índice que nos indique la posición en que se quedó para retomarla en las proximas vueltas, pero de esa manera dejariamos de fuera si el último elemento a buscar se encuentra en la primera posición}
    
\end{question}

%------------------------------------------------

\subsection{Pseudocodigo}

\begin{center}
  \begin{minipage}{0.7\linewidth} % Adjust the minipage width to accomodate for the length of algorithm lines
    \begin{algorithm}[H] 
      \SetKwInOut{Input}{entrada}\SetKwInOut{Output}{salida}
      \Input{$Arreglo_{1}, Arreglo_{2}$}
      \Output{$Arreglo_{3}$ (arreglo con repetidos)}
      \DontPrintSemicolon
      
      \caption{Encontrar elementos iguales}
      \label{alg:loop}
            {$CalcularLongitud(arreglo_{1})$}\\
            {$CalcularLongitud(arreglo_{2})$}\\
            {$lastIndex = 0$}\\
            \For{$i \gets 0$ hasta $longitud(arreglo_{1})$} {
              \For{$j \gets 0$ hasta $longitud(arreglo_{2})$} {
                \If{$arreglo_{1}[i]==arreglo_{2}[lastIndex]$}{
                  $arreglo_{3}.agregar(a[i])$\;
                  break\;
                }
                {$lastIndex = i + 1$}
              }
            }
    \end{algorithm}
  \end{minipage}
\end{center}

%------------------------------------------------

\newpage
\subsection{Implementación}

% File contents
\begin{file}[tarea1.cpp]
\begin{lstlisting}[language=C++]
#include <iostream>
#include <vector>

//Complejidad funcion principal O(n^2)
//por el doble for que recorre los arreglos
int main(){
  
  std::vector<int> a;
  a.push_back(2);
  a.push_back(5);
  a.push_back(5);
  a.push_back(5);

  std::vector<int> b;
  b.push_back(2);
  b.push_back(2);
  b.push_back(3);
  b.push_back(5);
  b.push_back(5);
  b.push_back(7);
  
  //vector de resultados
  std::vector<int>arr;

  int last_index = 0;

  //Usando fuerza bruta                    // O(n^2)
  for (int i = 0; i < a.size(); i++){      // O(n)
    for (int j = i; j < b.size(); j++){    // O(n)
      
      //std::cout << a[i] << "<-a comparacion b->"<< b[last_index] << "\n";
      
      if (a[i] == b[last_index]){
        arr.push_back(a[i]);
        break; // romper ciclo cuando sean iguales
      }
      
      last_index = i+1; //indice auxiliar para avanzar 
      
    }
  }
  
  //--------------------
  // Imprimir resultado
  //--------------------
  
  std::cout << "El resultado es: \n";       // 1
  for(int i = 0; i<arr.size(); i++){        // n
    std::cout << arr[i] << "\n";
  }
  
}

\end{lstlisting}
\end{file}
\href{https://github.com/luisballado/ADA/blob/main/practice_code/tarea1.cpp}{ver código en github}\\

Ejecutar desde una terminal

% Command-line "screenshot"
\begin{commandline}
	\begin{verbatim}
		$ g++ -o ./tarea1 ./tarea1.cpp
		$ ./tarea1
	\end{verbatim}
\end{commandline}



%----------------------------------------------------------------------------------------
\newpage
\subsection{Análisis de complejidad del mejor y peor caso}
\begin{warn}[MEJOR CASO:]
  encontrar todos los elementos del primer arreglo en el segundo arreglo de forma secuencial, sin el análisis de numeros que pertenezcan al primer arreglo ya que estarían ordenados $O(n)$
\end{warn}
\begin{warn}[PEOR CASO:]
  analizar todas las posibilidades y no encontrar similitudes $O(n^{2})$
\end{warn}

\end{document}

