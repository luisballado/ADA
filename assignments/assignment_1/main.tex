%%%%%%%%%%%%%%%%%%%%%%%%%%%%%%%%%%%%%%%%%
% Lachaise Assignment
% LaTeX Template
% Version 1.0 (26/6/2018)
%
% This template originates from:
% http://www.LaTeXTemplates.com
%
% Authors:
% Marion Lachaise & François Févotte
% Vel (vel@LaTeXTemplates.com)
%
% License:
% CC BY-NC-SA 3.0 (http://creativecommons.org/licenses/by-nc-sa/3.0/)
% 
%%%%%%%%%%%%%%%%%%%%%%%%%%%%%%%%%%%%%%%%%

%----------------------------------------------------------------------------------------
%	PACKAGES AND OTHER DOCUMENT CONFIGURATIONS
%----------------------------------------------------------------------------------------

\documentclass{article}

\input{structure.tex} % Include the file specifying the document structure and custom commands

%----------------------------------------------------------------------------------------
%	ASSIGNMENT INFORMATION
%----------------------------------------------------------------------------------------

\title{ITC-ADA-C1-2023: Assignment \#1} % Title of the assignment

\author{Luis Ballado\\ \texttt{luis.ballado@cinvestav.mx}} % Author name and email address

\date{CINVESTAV UNIDAD TAMAULIPAS --- \today} % University, school and/or department name(s) and a date

%----------------------------------------------------------------------------------------

\begin{document}

\maketitle % Print the title

%----------------------------------------------------------------------------------------
%	INTRODUCTION
%----------------------------------------------------------------------------------------

\section{Diseñe un algoritmo para encontrar todos los elementos comunes en dos listas ordenadas de números}

\begin{info} % Information block
  Se hace la prueba con los siguientes valores\\
  $2,5,5,5$ y $2,2,3,5,5,7$
\end{info}

\begin{question}
  \textbf{¿Cuál es el número máximo de comparaciones que realiza su algoritmo en función de las longitudes de las listas (m y n) respectivamente?}

  %se responde aquí
  \textbf{Máximo de comparaciones}
    
\end{question}

%------------------------------------------------

\subsection{Pseudocodigo}

\begin{center}
  \begin{minipage}{0.7\linewidth} % Adjust the minipage width to accomodate for the length of algorithm lines
    \begin{algorithm}[H] 
      \SetKwInOut{Input}{entrada}\SetKwInOut{Output}{salida}
      \Input{$Arreglo_{1}, Arreglo_{2}$}
      \Output{$Arreglo_{3}$ (arreglo con repetidos)}
      \DontPrintSemicolon
      
      \caption{Sum of Array Elements}
      \label{alg:loop}
            {$CalcularLongitud(arreglo_{1})$}\\
            {$CalcularLongitud(arreglo_{2})$}

            \For{$i \gets 0$ to $longitud(arreglo_{1})$} {
              \For{$j \gets 0$ to $longitud(arreglo_{2})$} {
                \If{$arreglo_{1}[i]==arreglo_{2}[j]$}{
                  \If{$!Existe(arr,arreglo_{1}[i]$)}{
                    agregar al arreglo de resultado a[i]\;
                  }
                }
                break;\;
              }
            }
    \end{algorithm}
  \end{minipage}
\end{center}

\begin{center}
  \begin{minipage}{0.7\linewidth} % Adjust the minipage width to accomodate for the length of algorithm lines
    \begin{algorithm}[H] 
      \SetKwInOut{Input}{input}\SetKwInOut{Output}{output}
      \Input{$arreglo_{3},valor$}
      \Output{$verdadero$ ó $falso$ (booleano si existe el elemento ó no)}
      \DontPrintSemicolon
      \caption{Función Existe}
      \label{alg:loop}
      \SetKwFunction{FLoop}{Existe}
      \SetKwProg{Fn}{Function}{}{}
      \BlankLine
      \Fn{\FLoop{$arreglo_{3},valor$}}{
        \For{$i \gets 0$ to $longitud(arreglo_{3})$} {
          \If{$arreglo_{3}[i]==valor$}{
            \KwRet {$Verdadero$}
          }
        }
        \KwRet {$Falso$}
      }
    \end{algorithm}
  \end{minipage}
\end{center}

%------------------------------------------------

\newpage
\subsection{Implementación}

% File contents
\begin{file}[tarea1.cpp]
\begin{lstlisting}[language=C++]
#include <iostream>
#include <vector>

bool existe(std::vector<int> _vector_, int valor){
  for(int i = 0; i < _vector_.size(); i++){ // n
    if (_vector_[i] == valor){              // 1
      return true;
    }
  }
  return false;
}

//Complejidad funcion principal O(n^2)
int main(){

  int a[] = {2,5,5,5};                     // 1
  int b[] = {2,2,3,5,5,7};                 // 1

  //vector de resultados
  std::vector<int>arr;                     // 1

  //longitud array1
  int a_length = sizeof(a)/sizeof(a[0]);   // 1
  //longitud array2
  int b_length = sizeof(b)/sizeof(b[0]);   // 1
  
  //Usando fuerza bruta                    // O(n^2)
  for (int i = 0; i < a_length; i++){      
    for (int j = 0; j < b_length; j++){    
      if (a[i] == b[j]){                   // 1
        //revisar si ya existe en el array
        if (!existe(arr,a[i])){            // 1
          arr.push_back(a[i]);
        }
        break;
      }
    }
  }

  // Imprimir resultado
  std::cout << "El resultado es: \n";       // 1
  for(int i = 0; i<arr.size(); i++){        // n
    std::cout << arr[i] << "\n";            // 1
  } 
}
\end{lstlisting}
\end{file}

Ejecutar desde una terminal

% Command-line "screenshot"
\begin{commandline}
	\begin{verbatim}
		$ g++ -o ./tarea1 ./tarea1.cpp
		$ ./tarea1
	\end{verbatim}
\end{commandline}



%----------------------------------------------------------------------------------------
\newpage
\subsection{Análisis de complejidad del mejor y peor caso}
\begin{warn}[MEJOR CASO:]
  Explicar aqui
\end{warn}
\begin{warn}[PEOR CASO:]
  Explicar aqui
\end{warn}

\end{document}

