%%%%%%%%%%%%%%%%%%%%%%%%%%%%%%%%%%%%%%%%%
% Lachaise Assignment
% LaTeX Template
% Version 1.0 (26/6/2018)
%
% This template originates from:
% http://www.LaTeXTemplates.com
%
% Authors:
% Marion Lachaise & François Févotte
% Vel (vel@LaTeXTemplates.com)
%
% License:
% CC BY-NC-SA 3.0 (http://creativecommons.org/licenses/by-nc-sa/3.0/)
% 
%%%%%%%%%%%%%%%%%%%%%%%%%%%%%%%%%%%%%%%%%

%----------------------------------------------------------------------------------------
%	PACKAGES AND OTHER DOCUMENT CONFIGURATIONS
%----------------------------------------------------------------------------------------

\documentclass{article}

\input{structure.tex} % Include the file specifying the document structure and custom commands

%----------------------------------------------------------------------------------------
%	ASSIGNMENT INFORMATION
%----------------------------------------------------------------------------------------

\title{ITC-ADA-C1-2023: Assignment \#3} % Title of the assignment

\author{Luis Ballado\\ \texttt{luis.ballado@cinvestav.mx}} % Author name and email address

\date{CINVESTAV UNIDAD TAMAULIPAS --- \today} % University, school and/or department name(s) and a date

%----------------------------------------------------------------------------------------

\begin{document}

\maketitle % Print the title

%----------------------------------------------------------------------------------------
%	INTRODUCTION
%----------------------------------------------------------------------------------------

\section{Considerando el algoritmo recursivo mostrado a continuación, responda las siguientes preguntas}

\begin{center}
  \begin{minipage}{0.7\linewidth} % Adjust the minipage width to accomodate for the length of algorithm lines
    \begin{algorithm}[H] 
      \SetKwInOut{Input}{entrada}\SetKwInOut{Output}{salida}
      \Input{Un arreglo $A[0..n-1]$ de números reales}
      \DontPrintSemicolon
      \caption{Algoritmo Misterio}
      \label{alg:loop}
      \If{n==1}{
        return A[0];
      }\Else{
        $temp \gets Misterio(A[0..n-2])$\\
        \If{temp <= A[n-1]}{
          return temp;
        }\Else{
          return A[n-1];
        }
      }
    \end{algorithm}
  \end{minipage}
\end{center}

\begin{question}
  \textbf{¿Qué calcula el algoritmo?}
  
  %se responde aquí
  \textit{}
    
\end{question}

\begin{question}
  \textbf{¿Cuál es el parámetro que indica el tamaño de la entrada del algoritmo?}
    %se responde aquí
  \textit{}
    
\end{question}

\begin{question}
  \textbf{¿Cuál es el parámetro que indica el tamaño de la entrada del algoritmo?}
    %se responde aquí
  \textit{}
    
\end{question}
\begin{question}
  \textbf{¿Cuál es el parámetro que indica el tamaño de la entrada del algoritmo?}
    %se responde aquí
  \textit{}
    
\end{question}
\begin{question}
  \textbf{¿Cuál es el parámetro que indica el tamaño de la entrada del algoritmo?}
    %se responde aquí
  \textit{}
    
\end{question}
\begin{question}
  \textbf{¿Cuál es el parámetro que indica el tamaño de la entrada del algoritmo?}
    %se responde aquí
  \textit{}
    
\end{question}

\begin{question}
  \textbf{¿Cuál es el parámetro que indica el tamaño de la entrada del algoritmo?}
    %se responde aquí
  \textit{}
    
\end{question}

\section{Dado el problema de encontrar el determinante de una matriz A de nxn , desarrolle los siguiente puntos:}

\begin{question}
  \textbf{Programe las versiones iterativa y recursiva del algoritmo para resolver el problema}
  %se responde aquí
  %\textit{Respuesta aqui}
\end{question}

\subsection{Implementación}
% File contents
\begin{file}[tarea1.cpp]
\begin{lstlisting}[language=C++]
#include <iostream>
#include <vector>

//Complejidad funcion principal O(n^2)
//por el doble for que recorre los arreglos
int main(){
  
  std::vector<int> a;
  a.push_back(2);
  a.push_back(5);
  a.push_back(5);
  a.push_back(5);

  std::vector<int> b;
  b.push_back(2);
  b.push_back(2);
  b.push_back(3);
  b.push_back(5);
  b.push_back(5);
  b.push_back(7);
  
  //vector de resultados
  std::vector<int>arr;

  int last_index = 0;

  //Usando fuerza bruta                    // O(n^2)
  for (int i = 0; i < a.size(); i++){      // O(n)
    for (int j = i; j < b.size(); j++){    // O(n)
      
      //std::cout << a[i] << "<-a comparacion b->"<< b[last_index] << "\n";
      
      if (a[i] == b[last_index]){
        arr.push_back(a[i]);
        break; // romper ciclo cuando sean iguales
      }
      
      last_index = i+1; //indice auxiliar para avanzar 
      
    }
  }
  
  //--------------------
  // Imprimir resultado
  //--------------------
  
  std::cout << "El resultado es: \n";       // 1
  for(int i = 0; i<arr.size(); i++){        // n
    std::cout << arr[i] << "\n";
  }
  
}

\end{lstlisting}
\end{file}
\href{https://github.com/luisballado/ADA/blob/main/practice_code/tarea1.cpp}{ver código en github}\\

Ejecutar desde una terminal

% Command-line "screenshot"
\begin{commandline}
	\begin{verbatim}
		$ g++ -o ./tarea1 ./tarea1.cpp
		$ ./tarea1
	\end{verbatim}
\end{commandline}

\begin{question}
  \textbf{Analice matemáticamente cada versión el algoritmo por separado usando las metodologías vistas en clase.}
    %se responde aquí
  \textit{Respuesta aqui}
\end{question}

\begin{question}
  \textbf{En base a los resultados obtenidos en el punto anterior determine cuál de los dos algoritmos es más eficiente}
    %se responde aquí
  \textit{Respuesta aqui}
\end{question}

\begin{question}
  \textbf{Para cada uno de los dos algoritmos desarrollados aplique el método descrito en el apartado "Doubling ratio experiments" del libro Algorithms de Sedgewick y Wayne, generando instancias de tamaños 1000,2000,etc; hasta lograr un radio de $2^{b}$, ejecutando 20 pruebas con cada tamaño y con cada algoritmo. Registre sus resultados.}
    %se responde aquí
  \textit{Respuesta aqui}
\end{question}

\begin{question}
  \textbf{Para cada algoritmo comparado realice una tabla con 5 predicciones, posteriores al tamaño con que se logró obtener el radio $2^{b}$}
    %se responde aquí
  \textit{Respuesta aqui}
\end{question}

\begin{question}
  \textbf{Para cada algoritmo comparado grafique los siguientes resultados de sus ejecuciones}
    %se responde aquí
  \textit{Respuesta aqui}
\end{question}

\begin{question}
  \textbf{Con base en los experimentos realizados y considerando un tiempo máximo de ejecución sobre su computadora de 7 días, ¿Cuál es el tamaño máximo de entrada que puede resolver cada algoritmo analizado?}
    %se responde aquí
  \textit{Respuesta aqui}
\end{question}

\begin{question}
  \textbf{Conclusiones respecto al orden de crecimiento de cada algoritmo observado empíricamente y constrástelas contra los resultados de sus análisis matemático}
    %se responde aquí
  \textit{Respuesta aqui}
\end{question}


\end{document}

